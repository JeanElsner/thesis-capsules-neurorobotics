\chapter{\abstractname}
The last few years have seen great strides being made in the fields of artificial intelligence and robotics. Many of the advancements are powered by the versatility of artificial neural networks. Especially in computer vision, deep learning architectures are particularly successful. For robotic systems, visual sensors are often their primary means to perceive the environment and negotiate it successfully. Thus computer vision systems form a vital part of the artificial intelligence necessary to allow autonomous operation of robots. There are however several factors limiting the widespread use of such systems in robotics. Namely, the need for massive data sets, the lack of generalization to novel configurations and the high computational cost associated with the training of neural networks. Neurorobotics tries to avoid these limitation by mimicking the behavior of biological systems, e.g. by using biologically inspired spiking neural networks. Recently, capsule networks have been proposed as an alternative biologically plausible computer vision system, albeit on a higher level of abstraction, based on second generation neural networks. Capsules are conceived as groups of neurons, that represent the presence of an entity and its instantiation parameters. It is claimed that by dynamically constructing a parse tree from a network of capsules, viewpoint invariance and better generalization can be achieved. In this thesis, object recognition tasks generated by the Neurorobotics Platform are used to evaluate the performance of capsule networks and quantify their ability to generalize compared to established methods such as convolutional and spiking neural networks.