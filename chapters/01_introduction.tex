\chapter{Introduction}\label{chapter:introduction}
The success of techniques from the field of artificial intelligence (AI) in recent years is quite evident. Not just are there ever more real world applications of AI found in search engines, security systems, industrial automation and more. But the term \emph{artificial intelligence} itself has become a vogue word\footnote{Often it is not immediately apparent what kind of technology is being referred to when the term AI is used. This is true to the point that opposing paradigms claim the mantle of AI. Classical AIs were conceived as systems performing logical inference on a knowledge database in the form of search algorithms (\emph{cognitivist} paradigm) as opposed to the current use of neural networks (an aspect of the \emph{emergent} paradigm).}, even outside of academic literature. A point can be made that much of the drive behind AI comes from the successful application of deep artificial neural networks in computer vision tasks \cite{paulun2018retinotopic}. As neural networks for computer vision tasks share similarities with the visual cortex \cite{lecun1998gradient}, a similar case can be made for the contribution of neuroscience: the eye is arguably the most intensely studied of the human sensory organs and the visual cortex is one the best understood parts of the brain. It is therefore only sensible to try and mimic the properties and behavior of brains when developing new systems. While cameras already surpass the spatial and temporal resolution of biological eyes, the human visual system remains unmatched in visual-cognitive tasks. Moreover, computer vision solutions currently available often have a high latency, precluding them from being used in real time experiments. In order to be able to validate biologically inspired AI models it is therefore necessary to simulate realistic sensory input \cite{falotico2017connecting}. The \emph{Neurorobotics Platform}\footnote{The Neurorobotics Platform is developed as a subproject of the \emph{Human Brain Project} (HBP). The HBP is a European Commission Future and Emerging Technologies Flagship Program intended to advance knowledge in the fields of neuroscience, computing and brain-related medicine over a period of 10 years.} was conceived to provide just such a simulation: a closed loop between a brain simulation based on artificial neural networks and a physics based simulation of robotic bodies embedded in a dynamic environment \cite{knoll2017neurorobotics}. In this thesis, sensor data as generated by an experiment within the Neurorobotics Platform was used to create challenging computer vision tasks. These tasks evaluate general object recognition performance and specific generalization and robustness, e.g. generalization to new viewpoints or robustness against occlusion.
\section{Motivation}
\section{Scope}

%The last few years have seen great strides being made in the fields of artificial intelligence and robotics. Many of the advancements are powered by the versatility of artificial neural networks.  Especially in computer vision, deep learning architectures are particularly successful. For robotic systems, visual sensors are often their primary means to perceive their environment and negotiate it successfully.  Thus computer vision systems form a vital part of the artificial intelligence necessary to allow autonomous operation of robots. There are however several factors limiting the widespread use of such systems in robotics. Namely, the need for massive data sets, the lack of generalization to novel configurations and the high computational cost associated with the training of neural networks. Neurorobotics tries to avoid these limitation by mimicking the behavior of biological systems, e.g. by using biologically inspired spiking neural networks. Recently,capsule networks have been suggested as an alternative biologically plausible computer vision system, albeit on a higher level of abstraction, based on second generation neural networks. Capsules are conceived as groups of neurons, that represent the presence of an entity and its instantiation parameters. It is proposed that by dynamically constructing a parse tree from a network of capsules, viewpoint invariance and better generalization can be achieved. In this thesis, object recognition tasks generated by the Neurorobotics Platform are used to evaluate the performance of capsule networks and quantify their ability to generalize compared to established methods such as convolutional and spiking neural networks.x