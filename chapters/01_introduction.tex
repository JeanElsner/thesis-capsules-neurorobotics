\chapter{Introduction}\label{chapter:introduction}
The success of techniques from the field of artificial intelligence (AI) in recent years is quite evident. Not just are there ever more real-world applications of AI found in search engines, security systems, industrial automation and more. But the term \emph{artificial intelligence} itself has become a vogue word, even outside of academic literature. In fact, often it is not immediately apparent what kind of technology is being referred to when the term AI is used. This is true to the point that opposing paradigms claim the mantle of AI. Classical AIs were conceived as systems performing logical inference on a knowledge database in the form of search algorithms (\emph{cognitivist} paradigm) as opposed to the current use of artificial neural networks (an aspect of the \emph{emergent} paradigm). Research into artificial neural networks today has diverged into two distinct branches \cite{krichmar2018neurorobotics}. On the one hand there are spiking neural networks that are mostly used as tools to study biological nerve cells by closely modeling the physical properties of neurons and synapses. On the other there are multi-layer perceptrons, convolutional neural networks chief among them, that are optimized for machine learning tasks such as object detection in images or speech recognition. Spiking neural network have inherently great potential, as they are directly derived from biological examples. Still they are almost always outperformed by the latter \cite{rosenblatt1958perceptron,rumelhart1985learning}. A point can be made that much of the drive behind AI comes from the successful application of deep artificial neural networks in computer vision tasks \cite{paulun2018retinotopic}. As convolutional neural networks designed for computer vision share similarities with the mammalian visual cortex \cite{lecun1998gradient}, a similar case can be made for the contribution of neuroscience. The eye is arguably the most intensely studied of the human sensory organs and the visual cortex is one of the best understood parts of the brain. It is therefore only sensible to try and mimic the properties and behavior of brains when developing new vision systems. While cameras already surpass the spatial and temporal resolution of biological eyes, the human visual system remains unmatched in visual-cognitive tasks. Moreover, computer vision solutions currently available often have a high latency, precluding them from being used in real time experiments. In order to be able to validate biologically inspired AI models it is therefore necessary to simulate a realistic environment \cite{falotico2017connecting}. The Human Brain Project's \emph{Neurorobotics Platform}\footnote{Available at \url{http://neurorobotics.net/}} was conceived to provide just such a simulation, a closed loop between a brain simulation based on artificial neural networks and a physics-based simulation of robotic bodies embedded in a dynamic environment \cite{knoll2017neurorobotics}. In this thesis, sensor data as generated by an experiment within the Neurorobotics Platform was used to create challenging computer vision tasks. These tasks evaluate general object recognition performance and specific generalization and robustness, e.g. generalization to new viewpoints or robustness against occlusion. The use of a high-fidelity simulation platform allows for the generation of big labelled datasets that can be fine-tuned towards the strengths and weaknesses of the architecture at hand. Something that would otherwise require many hours of manual labor.
\section{Motivation}
Capsule networks have been proposed as a more biologically plausible alternative to convolutional neural network in computer vision. The routing of signals between capsule layers, based on grouping votes from the lower layer, allows capsules to encode the image structure into more sophisticated internal representations. To address the open question of whether this leads to better generalization, capsule network architectures will be explicitly evaluated in this thesis using custom-made datasets created within the Neurorobotics Platform. Networks representing both ends of the biological plausibility spectrum will be used as baseline to see where capsule networks fit in performance-wise and whether there is a need for them at all.
\section{Scope}
The thesis starts with a derivation of the state of the art of the artificial neural networks used as baseline in this thesis in chapter \ref{chapter:state-of-the-art}. The chapter also includes details about the limitations of the discussed architectures and explains classification by example of object recognition. This is followed by an introduction of the motivation and theory behind capsule networks as well as a brief review of their development in chapter \ref{chapter:capsules}. How the Neurorobotics Platform was used to create the datasets as well as how exactly each architecture was trained and tested and what metrics were applied is explained in chapter \ref{chapter:experimental-setup}. The numerical results from the experiments described in the previous chapter are presented in chapter \ref{chapter:results} and interpreted in a discussion found in chapter \ref{chapter:discussion}. Finally, the thesis concludes with an outlook on possible further developments and open questions in chapter \ref{chapter:conclusion}.

%The last few years have seen great strides being made in the fields of artificial intelligence and robotics. Many of the advancements are powered by the versatility of artificial neural networks.  Especially in computer vision, deep learning architectures are particularly successful. For robotic systems, visual sensors are often their primary means to perceive their environment and negotiate it successfully.  Thus computer vision systems form a vital part of the artificial intelligence necessary to allow autonomous operation of robots. There are however several factors limiting the widespread use of such systems in robotics. Namely, the need for massive data sets, the lack of generalization to novel configurations and the high computational cost associated with the training of neural networks. Neurorobotics tries to avoid these limitation by mimicking the behavior of biological systems, e.g. by using biologically inspired spiking neural networks. Recently,capsule networks have been suggested as an alternative biologically plausible computer vision system, albeit on a higher level of abstraction, based on second generation neural networks. Capsules are conceived as groups of neurons, that represent the presence of an entity and its instantiation parameters. It is proposed that by dynamically constructing a parse tree from a network of capsules, viewpoint invariance and better generalization can be achieved. In this thesis, object recognition tasks generated by the Neurorobotics Platform are used to evaluate the performance of capsule networks and quantify their ability to generalize compared to established methods such as convolutional and spiking neural networks.x